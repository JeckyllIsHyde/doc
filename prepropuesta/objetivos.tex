\mode*
\section[Objetivos]{Objetivo general y objetivos espec\'ificos}
\label{sec:objs}

\mode<presentation>{
  \begin{frame}[label=objetivos]
    \transduration{1}
    \frametitle{Qu\'e se busca con esta investigaci\'on?}
    \begin{center}
      \LARGE \textbf{\textcolor{blueun}{Objetivo General y Objetivos Espec\'ificos}}
    \end{center}  
  \end{frame}
}
\subsection[General]{Objetivo general}
\label{sec:objgen}
\mode<presentation>{
  \begin{frame}<1-2>[label=def_objetivos]
    \transduration<1->{5}
    \frametitle{Qu\'e se busca con esta investigaci\'on?}
    \textbf{\textcolor{blueun}{OG:}} Desarrollar un \alert<2>{marco de experiementaci\'on}\only<2-2>{\footnote{Marco o framework de investigaci\'on}}, mediante el cual se pueda implementar las hip\'otesis de soluci\'on de \alert<3>{los problemas de locomoci\'on}\only<3-3>{\footnote{Problemas actuales de robotica b\'ipeda}} de la rob\'otica subactuada y de caminadores.\\
    \hyperlink<2>{def_framework}{\beamergotobutton{Definici\'on de Framework}}
    \hyperlink<3>{def_problema}{\beamergotobutton{Problemas de locomoci\'on}}
  \end{frame}
  % Marco de Experimentaci\'on:
  % Es un aparato de conocimiento din\'amico con ejemplos construidos de plataformas b\'ipedas rob\'oticas que siguen los siguientes pasos
  % 1) Selecci\'on de plataforma: Que se quiere investigar o proponer? Que se quiere resolver? Que se quiere mejorar?
  % 2) Dise\~no mecanico: Planos, ensambles, elementos estandar, tipo de fabricaci\'on y presupuesto aproximado.
  % 3) Dise\~no electronico: Esquem\'aticos, PCBs(DIP,SMT), selecci\'on de microprecesadores, microcontroladores, sensores, actuadores y presupuesto aproximado.
  % 4) Sistema distribuido: Implementaci\'on red de sensores-actuadores, linux embebido, puesta en marcha y configuraci\'on del sistema distruibuido.
  % 5) Capa de aplicaciones: Selecci\'on de librer\'ias, implementaci\'on en un lenguaje de alto nivel de las estrategias de control y cognicion.
  \begin{frame}[plain,t,label=def_framework]
    \transduration<1>{1}
    \transduration<2>{2}
    \transduration<3->{3}
    \hspace*{-0.8cm}\parbox[t]{\textwidth}{
      \only<1->{\vspace*{-0.4cm}\hspace*{-1.5cm}
        \colorbox{blueun}{
          \parbox[t][1.5cm][c]{\paperwidth}{
            \textcolor{white}{\Large\quad{MARCO DE EXPERIMENTACI\'ON}}
          }
        }
      }\vspace{0.05cm}\\
      \only<2>{\alert<2>{Framework o marco de experimentaci\'on se define en esta propuesta como:}}\\
      \only<3->{\alert<3>{Un aparato de conocimiento din\'amico con ejemplos construidos de plataformas b\'ipedas rob\'oticas que siguen los siguientes pasos.}}
      \only<3->{\small
        \begin{columns}[t] %¡Columnas como en TeX normal!
          \begin{column}{7cm}
            \begin{enumerate}
            \item<4-|alert@5> \hyperlink{def_framework<5>}{\textbf{Selecci\'on de plataforma\only<5>{:}} \only<5>{\scriptsize Qu\'e se quiere investigar o proponer? Qu\'e se quiere resolver? Qu\'e se quere mejorar?}}
            \item<4-|alert@6> \hyperlink{def_framework<6>}{\textbf{Dise\~no mec\'anico\only<6>{:}} \only<6>{\scriptsize Planos, ensamples, elementos estandar, tipo de fabricaci\'on y presupuesto aproximado.}}
            \item<4-|alert@7> \hyperlink{def_framework<7>}{\textbf{Dise\~no electr\'onico\only<7>{:}} \only<7>{\scriptsize Esquem\'aticos, ensambles, elementos estandar, tipo de fabricaci\'on y presupuesto aproximado.}}
            \item<4-|alert@8> \hyperlink{def_framework<8>}{\textbf{Sistema embebido y distribuido\only<8>{:}} \only<8>{\scriptsize Implementaci\'on de red, GNU-Linux embebido, puesta en marcha y configuraci\'on.}}
            \item<4-|alert@9> \hyperlink{def_framework<9>}{\textbf{Aplicaciones\only<9>{:}} \only<9>{\scriptsize Selecci\'on de librerias, lenguaje de alto nivel, implementaci\'on estrategias de control y congnici\'on.}}
            \end{enumerate}
          \end{column}
          \begin{column}{3cm}
            \\% espacio para ajustar las graficas
            \includegraphics<5>[height=4cm]{../images/DENISE.png}
            \includegraphics<6>[height=4cm]{../images/DLR-BIPED.png}
            \includegraphics<7>[height=4cm]{../images/ASIMO.png}
            \includegraphics<8>[height=4cm]{../images/RABBIT.png}
            \includegraphics<9>[height=4cm]{../images/TODDLE-MIT.png}
          \end{column}
        \end{columns}
      }\vspace{0.5cm}
      \hyperlink<4-10>{def_objetivos<2>}{\beamerreturnbutton{Volver al Objetivo General}}
    }
  \end{frame}
  \againframe<3>{def_objetivos}
  \begin{frame}[label=objgen]
    \transduration<1->{5}
    \frametitle{Qu\'e se quiere lograr con esta investigaci\'on?}
    \framesubtitle{Framework de investigaci\'on y desarrollo de rob\'otica b\'ipeda}
    \begin{center}
      \includegraphics[height=7.0cm]{../images/objGen.png}
    \end{center}
  \end{frame}
}
El principal objetivo de esta investigaci\'on se describe de forma global y gr\'afica en la Figura \ref{fig:objGen}, y es enunciado a continuaci\'on:\par
\textbf{OG:} Dise\~nar y construir un \emph{marco de experimentaci\'on}\footnote{compuesto por el dise\~no de una red de sensores-actuadores distribuida y un conjunto de dise\~no de eslanbones y articulaciones modulares para ser frabricados mediante prototipado r\'apido} mediante el cual se pueda implementar\footnote{estudiar, proponer y/o comprobar} las hip\'otesis que solucionen los problemas de locomoci\'on de la rob\'otica subactuada y de caminadores\footnote{que permita la b\'usqueda de estructuras mecatr\'onicas eficientes energ\'eticamente, capaces de lograr la locomoci\'on requerida por las necesidades fundamentales de los robots caminadores}.\par
\begin{figure}[!htb]
  \centering
  \includegraphics[scale=0.5]{../images/objGen.png}
  \caption{Objetivo General}
  \label{fig:objGen}
\end{figure}

\subsection[Especificos]{Objetivos espec\'ificos}
\label{sec:objesp}
\mode<presentation>{
  \begin{frame}[label=def_objesp]
    \transduration{5}
    \frametitle{Qu\'e se requiere y qu\'e se obtiene para el Obj. Gen?}
    \framesubtitle{Algunos logros requeridos para el objetivo principal}
    \begin{center}
      \hyperlink{exp_capas<2>}{\includegraphics[height=2.0cm]{../images/objCapaFisica.png}}
      \hyperlink{exp_capas<3>}{\includegraphics[height=2.0cm]{../images/objCapaRAS.png}}\\
      \hyperlink{exp_capas<4>}{\includegraphics[height=2.0cm]{../images/objCapaControl.png}}
      \hyperlink{exp_capas<5>}{\includegraphics[height=2.0cm]{../images/objCapaCognitiva.png}}
    \end{center}
    \hyperlink{exp_capas}{\beamergotobutton{Explicaci\'on de Capas}}
  \end{frame}
}
\begin{enumerate}[\textbf{OE:} 1.]
\item Modelar, simular, analizar y sintetizar mecanismos subactuados que optimicen la energ\'ia para la locomoci\'on de caminar, saltar o correr.\par 
\item Dise\~nar y construir una plataforma rob\'otica modular modular y para prototipado (ver Figura \ref{fig:objCapas} Capa Física), capaz de configurar cadenas cinem\'aticas  controladas y/o monitoreadas bajo el sistema distribuido.\par
\item Implementar un sistema distribuido de sensores y actuadores que funcione en tiempo-real  (ver Figura \ref{fig:objCapas} Capa de sensorial y de Actuaci\'on), bajo el principio de manejo por disparo-de-eventos y/o manejo por disparo-por-tiempos\cite{Kimm2012}.\par
\item Dise\~nar, simular e implementar diferentes controles de locomoci\'on (ver Figura \ref{fig:objCapas} Capa de control de locomoci\'on) inspirados en las nuevas tendencias de investigaci\'on de caminadores sobre plataformas rob\'oticas modulares construidas.\par
\item Dise\~nar, simular e implementar estrategias y actividades colaborativas usando una red de caminadores  (ver Figura \ref{fig:objCapas} Capa Cognitiva.).\par
\end{enumerate}
\begin{figure}[!htb]
  \centering
  \includegraphics[scale=0.45]{../images/objCapaFisica.png}
  \includegraphics[scale=0.45]{../images/objCapaRAS.png}\\
  \includegraphics[scale=0.45]{../images/objCapaControl.png}
  \includegraphics[scale=0.45]{../images/objCapaCognitiva.png}
  \caption{Objetivos Específicos}
  \label{fig:objCapas}
\end{figure}
