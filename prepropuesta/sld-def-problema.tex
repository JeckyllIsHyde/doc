\mode*
% Problemas de la robotica b\'ipeda:
% Los problemas que se intentar\'a resolver son relacionados con el uso eficiente de la energ\'ia, la din\'amica pasiva de caminadores y forman parte activa de la investigaci\'on.
% 1) Estructuras mec\'anicas: Dise\~no de rodillas, tobillos, talon-planta-antepie, exploraci\'on de estructuras paralelas, SLIP, Curved-beams hopping robots. An\'alisis y s\'intesis para el torso.
% 2) Capa sensora y de actuaci\'on: Reducir peso y complejidad del cableado, asegurar ancho de banda, asegurar latencias y escabilidad de la red.
% 3) Control: Asegurar estabilidad y convergencia a los ciclos limite adem\'as auto-aprendizaje, robustez ante perturbaciones, generaci\'on On-Line, mediante Central patern generator, Redes neuronales y Logica difusa, aprendizaje de m\'aquina, LQR-Trees y Lyapunov.
% 4) Planeaci\'on y generaci\'on de trayectorias: Optimizando la energ\'ia seg\'un el sistema mecatr\'onico proponer las trayectorias, evasi\'on de obst\'aculos. Algoritmos evolutiovos como GAs y PSOs.
% 5) Capa de cognitiva: Agentes inteligentes, estrategias conjuntas y colaborativas.
\begin{frame}[plain,t,label=def_problema]
  \hspace*{-0.8cm}\parbox[t]{\textwidth}{
    \only<1->{\vspace*{-0.4cm}\hspace*{-1.5cm}
      \colorbox{blueun}{
        \parbox[t][1.5cm][c]{\paperwidth}{
          \textcolor{white}{\Large\quad{PROBLEMAS DE LA ROBOTICA B\'IPEDA}}
        }
      }
    }
    \only<2-3>{\alert<2>{Los problemas que se intentar\'a resolver, comprobar o investigar son relacionados con el uso eficiente de la energ\'ia, la din\'amica pasiva de caminadores, los m\'etodos bio-inspirados y el aprendizaje de m\'aquina, adem\'as formar\'an parte activa de esta investigaci\'on.}}\\
    \only<3->{\alert<3>{Se clasifican seg\'un la capa del framework de investigaci\'on:}}
    \only<3->{\small
      \begin{columns}[t] %¡Columnas como en TeX normal!
        \begin{column}{3cm}
          \\% espacio para ajustar las graficas
          \includegraphics<5>[height=4cm]{../images/DisenoMecanico.png}
          \includegraphics<6>[height=4cm]{../images/DisenoMecanico.png}
          \includegraphics<7>[height=4cm]{../images/DisenoMecanico.png}
          \includegraphics<8>[height=4cm]{../images/DisenoMecanico.png}
          \includegraphics<9>[height=4cm]{../images/DisenoMecanico.png}
        \end{column}
        \begin{column}{7cm}
          \begin{enumerate}
          \item<4-|alert@5> \hyperlink{def_problema<5>}{\textbf{Estructuras mec\'anicas\only<5>{:}} \only<5>{\scriptsize Dise\~no de rodillas, tobillos, talon-planta-antepie, exploraci\'on de estructuras paralelas, SLIP, Curved-beams hopping robots. An\'alisis y s\'intesis para el torso.}}
          \item<4-|alert@6> \hyperlink{def_problema<6>}{\textbf{Capa sensora y de actuaci\'on\only<6>{:}} \only<6>{\scriptsize Reducir peso y complejidad del cableado, asegurar ancho de banda, asegurar latencias y escabilidad de la red.}}
          \item<4-|alert@7> \hyperlink{def_problema<7>}{\textbf{Control\only<7>{:}} \only<7>{\scriptsize Asegurar estabilidad y convergencia a los ciclos limite adem\'as auto-aprendizaje, robustez ante perturbaciones, generaci\'on On-Line, mediante Central patern generator, Redes neuronales y Logica difusa, aprendizaje de m\'aquina, LQR-Trees y Lyapunov.}}
          \item<4-|alert@8> \hyperlink{def_problema<8>}{\textbf{Trayectorias\only<8>{:}} \only<8>{\scriptsize Optimizando la energ\'ia seg\'un el sistema mecatr\'onico proponer las trayectorias, evasi\'on de obst\'aculos. Algoritmos evolutiovos como GAs y PSOs.}}
          \item<4-|alert@9> \hyperlink{def_problema<9>}{\textbf{Capa de cognici\'on\only<9>{:}} \only<9>{\scriptsize Agentes inteligentes, estrategias conjuntas y colaborativas.}}
          \end{enumerate}
        \end{column}
      \end{columns}
    }\vspace{0.5cm}
    \hyperlink<4-10>{def_objetivos<3>}{\beamerreturnbutton{Volver al Objetivo General}}
  }
\end{frame}
